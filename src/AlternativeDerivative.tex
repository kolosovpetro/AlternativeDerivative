\documentclass[12pt,letterpaper,oneside,reqno]{amsart}
\usepackage{amsfonts}
\usepackage{amsmath}
\usepackage{amssymb}
\usepackage{amsthm}
\usepackage{float}
\usepackage{mathrsfs}
\usepackage{colonequals}
\usepackage[font=small,labelfont=bf]{caption}
\usepackage[left=1in,right=1in,bottom=1in,top=1in]{geometry}
\usepackage[pdfpagelabels,hyperindex,colorlinks=true,linkcolor=blue,urlcolor=magenta,citecolor=green]{hyperref}
\usepackage{setspace}
\onehalfspacing
\emergencystretch=1em

\newcommand \coeffA [3][A] {{\mathbf{#1}} \sb{#2,#3}}

\newtheorem{thm}{Theorem}[section]
\newtheorem{example}[thm]{Example}
\newtheorem{definition}[thm]{Definition}
\newtheorem{notation}[thm]{Notation}

\title[Another approach to get derivative of odd-power]
{Another approach to get derivative of odd-power}
\author[Petro Kolosov]{Petro Kolosov}
\email{kolosovp94@gmail.com}
\keywords{
    Partial differential equations, PDE, Exponential function
}
\urladdr{https://razumovsky.me/}
\subjclass[2010]{32W50, 33B10}
\date{\today}
\hypersetup{
    pdftitle={An exponential identity in terms of partial derivatives},
    pdfsubject={
        Partial differential equations, PDE, Exponential function
    },
    pdfauthor={Petro Kolosov},
    pdfkeywords={
        Partial differential equations, PDE, Exponential function
    }
}
\begin{document}
    \begin{abstract}
        This manuscript provides another approach to get derivative of odd-power, that is approach based on partial derivatives
of the polynomial function $f_y$ defined as
\[
    f_{y} (x, z) = \sum_{k=1}^{z} \sum_{r=0}^{y} \mathbf{A}_{y,r} k^r (x-k)^r
\]
where $x, z\in \mathbb{R}$, $y$ is fixed constant $y \in \mathbb{N}$ and $\mathbf{A}_{y,r}$ are real coefficients.
    \end{abstract}

    \maketitle

    \tableofcontents


    \section{Introduction and Main Results} \label{sec:introduction}
    This manuscript provides an exponential identity in terms of partial derivatives,
extending the main idea explained in~\cite{kolosov_2022} that gives polynomial identity in a form as follows
\begin{equation}
    n^{2m+1} = \sum_{k=1}^{n} \sum_{r=0}^{m} \coeffA{m}{r} k^r (n-k)^r, \quad (m,n) \in \mathbb{N},
    \label{eq:odd-power-identity}
\end{equation}
where $\coeffA{m}{r}$ are real coefficients defined recursively, see~\cite{kolosov2016link}.
Define the function $f$ such that based on the identity~\eqref{eq:odd-power-identity} with the only difference that
values of $n, m$ in its left part appear to be parameters of the function $f$, that is
\begin{definition}
    \begin{equation}
        f(x, y, z) = \sum_{k=1}^{z} \sum_{r=0}^{y} \coeffA{y}{r} k^r (x-k)^r\label{eq:definition-f}
    \end{equation}
\end{definition}


    \section{Conclusions}\label{sec:conclusions}
    In this manuscript, we have reviewed an approach to get derivative of odd-power using identity in partial derivatives
of the function $f$ evaluated at fixed point $(u,u) \in \mathbb{R}^2.$
Main results of the manuscript can be validated using Mathematica programs available online at~\cite{kolosov2022another}.

    \bibliographystyle{alpha}
    \bibliography{AlternativeDerivativeReferences}

\end{document}
