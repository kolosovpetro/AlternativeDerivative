\begin{example}
    \normalfont
    Theorem~\ref{thm:main-theorem} example for $x\in\mathbb{R}, \; z\in \mathbb{R}$ and $y=1$.
    Consider the explicit form of the function $f_{1} (x, z)$ i.e.,
    \[
        f_1(x, z) = 3 x z - 3 z^2 + 3 x z^2 - 2 z^3
    \]
    Therefore, derivative of $f_{1}$ with respect to $x$ equals to
    \[
        (f_1)^{'}_{x} = \lim_{d \to 0} \frac{3 d z + 3 d z^2}{d} = 3 z + 3 z^2
    \]
    Consider derivative of the function $f_1$ with respect to $z$, that is
    \begin{align*}
    (f_1)
        ^{'}_{z}
        &= \lim_{d \to 0} \left[\frac{-3 d^2 - 2 d^3 + 3 d x + 3 d^2 x - 6 d z - 6 d^2 z + 6 d x z - 6 d z^2}{d} \right] \\
        &= \lim_{d \to 0} \left[ -3 d - 2 d^2 + 3 x + 3 d x - 6 z - 6 d z + 6 x z - 6 z^2 \right] \\
        &=3 x - 6 z + 6 x z - 6 z^2
    \end{align*}
    Summing up both partial derivatives $(f_1)^{'}_{x}$ and $(f_1)^{'}_{z}$, we get
    \begin{align*}
    (f_1)
        ^{'}_{x} + (f_1)^{'}_{z} = 3 x - 3 z + 6 x z - 3 z^2
    \end{align*}
    Evaluating in point $(u, u)$ yields
    \begin{align*}
        \frac{d}{dt} t^{3} (u) = [(f_1)^{'}_{x} + (f_1)^{'}_{z}] (u,u)  = 3 u^2
    \end{align*}
    That confirms the results of the theorem~\ref{thm:main-theorem}.
\end{example}