This manuscript provides an exponential identity in terms of partial derivatives,
extending the main idea explained in~\cite{kolosov_2022} that gives polynomial identity in a form as follows
\begin{equation}
    n^{2m+1} = \sum_{k=1}^{n} \sum_{r=0}^{m} \coeffA{m}{r} k^r (n-k)^r, \quad (m,n) \in \mathbb{N},
    \label{eq:odd-power-identity}
\end{equation}
where $\coeffA{m}{r}$ are real coefficients defined recursively, see~\cite{kolosov2016link}.
Define the function $f$ such that based on the identity~\eqref{eq:odd-power-identity} with the only difference that
values of $n, m$ in its left part appear to be parameters of the function $f$, that is
\begin{definition}
    \begin{equation}
        f(x, y, z) = \sum_{k=1}^{z} \sum_{r=0}^{y} \coeffA{y}{r} k^r (x-k)^r\label{eq:definition-f}
    \end{equation}
\end{definition}
Important to note that upper bound of the sum $\sum_{k=1}^{z}$ in~\eqref{eq:definition-f} is parameter of the function $f$ in contrast
to the equation~\eqref{eq:odd-power-identity} where upper bound of the sum is $n$.
At first glance, equation~\eqref{eq:definition-f} might look complex and not immediately understood, so in order to understand
the function $f$ and polynomials it produces.
Substituting the values of $y=1,2,3$ we get the following polynomials
\begin{align*}
    f(x, 1, z) &= 3 x z - 3 z^2 + 3 x z^2 - 2 z^3 \\
    f(x, 2, z) &= 5 x^2 z - 15 x z^2 + 15 x^2 z^2 + 10 z^3 - 30 x z^3 + 10 x^2 z^3 +
    15 z^4 - 15 x z^4 + 6 z^5 \\
    f(x, 3, z) &= -7 x z + 14 x^2 z + 7 z^2 - 42 x z^2 + 35 x^3 z^2 + 28 z^3 - 140 x^2 z^3 + 70 x^3 z^3 + 175 x z^4 \\
    &- 210 x^2 z^4 + 35 x^3 z^4 - 70 z^5 + 210 x z^5 - 84 x^2 z^5 - 70 z^6 + 70 x z^6 - 20 z^7
\end{align*}
According to the main topic of the current manuscript, it provides an odd-exponential identity
in terms of partial derivatives.
Therefore, define the exponential function we work in context of.
Exponential function $g$ is a function of two variables defined as follows
\begin{definition}(Exponential function.)
    \begin{equation}
        g(x, y) = x^{2y + 1}, \quad (x,y) \in \mathbb{R}
        \label{eq:definition-g}
    \end{equation}
\end{definition}
One more important thing to conclude on is to conclude on partial derivative notation,
more precisely the following notation for the partial derivative is used across the manuscript and remains unchanged
\begin{notation} (Partial derivative.) Let be a function $f(x_1, x_2, \dots, x_n)$ defined over the real space $\mathbb{R}^n$.
We denote partial derivative of the function $f$ with respect to $x_i, \quad 1 \leq i \leq n$ as follows
    \begin{equation*}
        f^{'}_{x_i} = \lim_{\Delta x_i \to 0} \frac{f(x_1, x_2, \dots, x_i + \Delta x_i, \dots, x_n) - f(x_1, x_2, \dots, x_n)}{\Delta x_i}
    \end{equation*}
Derivative of the function $f$ with respect to $x_i, \quad 1 \leq i \leq n$ evaluated in point $(y_1, y_2, \dots, y_n)$ is denoted as follows
    \begin{equation*}
        f^{'}_{x_i} (y_1, y_2, \dots, y_n)
    \end{equation*}
\end{notation}


