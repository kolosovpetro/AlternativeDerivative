This manuscript provides another approach to get derivative of odd-power, that is approach based on partial derivatives
of the polynomial function $f_y(x,z)$ defined as
\[
    f_{y} (x, z) = \sum_{k=1}^{z} \sum_{r=0}^{y} \mathbf{A}_{y,r} k^r (x-k)^r
\]
where $x, z\in \mathbb{R}$, $y$ is fixed constant $y \in \mathbb{N}$ and $\mathbf{A}_{y,r}$ are real coefficients.
So that we discuss another approach to get derivative of odd-power, that is an approach based
on identity in terms of sum of partial derivatives of $f_{y}$.
The function $f_{y}$ is based on the main results of the manuscript~\cite{kolosov_2022}
that explains an odd-power in a form as follows
\begin{equation}
    n^{2m+1} = \sum_{k=1}^{n} \sum_{r=0}^{m} \coeffA{m}{r} k^r (n-k)^r
    \label{eq:odd-power-identity}
\end{equation}
where $m$ is fixed constant $m\in\mathbb{N}$, $n \in \mathbb{N}$ and $\coeffA{m}{r}$ are real coefficients defined
recursively, see~\cite{kolosov2016link}.
We define the function $f_{y}$ such that based on the identity~\eqref{eq:odd-power-identity}
with the only difference that values of $n, m$ in the right part of~\eqref{eq:odd-power-identity}
appear to be parameters of the function $f_{y}$.
In contrast to the equation~\eqref{eq:odd-power-identity}, upper bound $n$ of the sum $\sum_{k=1}^{n}$ turned into fixed
function's parameter $y$ as well.
Let the function $f_{y}$ be defined as follows
\begin{definition} (Polynomial function $f_{y}$.)
    \begin{equation}
        f_{y} (x, z) = \sum_{k=1}^{z} \sum_{r=0}^{y} \coeffA{y}{r} k^r (x-k)^r
        \label{eq:definition-f}
    \end{equation}
\end{definition}
where $x, z\in \mathbb{R}$ and $y$ is constant $y \in \mathbb{N}$.
Note that for every $x\in\mathbb{R}$ and constant $y\in\mathbb{N}$ the polynomial identity satisfies
\begin{equation*}
    f_{y} (x, x) = x^{2y+1}
\end{equation*}
At first glance, equation~\eqref{eq:definition-f} might look complex, so in order to clarify
the function $f_y$ and polynomials it produces, let there be a few examples.
Substituting the values of $y=1,2,3$ to the function $f_y$ we get the following polynomials in $x, z$
\begin{align*}
    f_{1} (x, z) &= 3 x z - 3 z^2 + 3 x z^2 - 2 z^3 \\
    f_{2} (x, z) &= 5 x^2 z - 15 x z^2 + 15 x^2 z^2 + 10 z^3 - 30 x z^3 + 10 x^2 z^3 +
    15 z^4 - 15 x z^4 + 6 z^5 \\
    f_{3} (x, z) &= -7 x z + 14 x^2 z + 7 z^2 - 42 x z^2 + 35 x^3 z^2 + 28 z^3 - 140 x^2 z^3 + 70 x^3 z^3 + 175 x z^4 \\
    &- 210 x^2 z^4 + 35 x^3 z^4 - 70 z^5 + 210 x z^5 - 84 x^2 z^5 - 70 z^6 + 70 x z^6 - 20 z^7
\end{align*}
These polynomials are obtained by rearranging the sums in the definition~\eqref{eq:definition-f} as
\[
    f_{y} (x, z) = \sum_{r=0}^{y} \coeffA{y}{r} \left[ \sum_{k=1}^{z} k^r (x-k)^r \right]
\]
So that part $\sum_{k=1}^{z} k^r (x-k)^r$ is polynomial
in $x, z$ calculated using Faulhaber's formula~\cite{beardon1996sums}.
According to the main topic of the current manuscript, it provides another approach to get derivative of odd-power.
Therefore, we define odd-power function we work in the context of.
The odd-power function $g_y$ is a function defined as follows
\begin{definition}(Odd-power function $g_y$.)
    \begin{equation*}
        g_{y}(x) = x^{2y + 1}
    \end{equation*}
\end{definition}
where $x\in \mathbb{R}$ and $y$ is constant $y\in \mathbb{N}$.
The Interesting part is that odd-power function $g_{y} (x)$ may be obtained
as a partial case of the function $f_y$ for $z=x$.
Also, the ordinary derivative of odd-power $\frac{d}{dx} g_{y}$ evaluate in fixed point $u\in\mathbb{R}$
may be obtained as a sum of partial derivatives of $f_y$ evaluated in fixed point $(u,u)$.
We explain this further in the manuscript.
One more important thing remains to conclude is to define partial derivative's notation.
More precisely, the following notation for partial derivatives
is used across the manuscript and remains unchanged
\begin{notation} (Partial derivative.)
    Let be a function $f(x_1, x_2, \dots, x_n)$ defined over the real space $\mathbb{R}^n$.
    We denote partial derivative of the function $f$ with respect to $x_i$ as follows
    \begin{equation*}
        f^{'}_{x_i} = \lim_{\Delta x_i \to 0} \frac{f(x_1, x_2, \dots, x_i + \Delta x_i, \dots, x_n) - f(x_1, x_2, \dots, x_n)}{\Delta x_i}
    \end{equation*}
\end{notation}
Partial derivative of the function $f_{x_i}$ with respect to $x_i$
evaluate in point $(y_1, y_2, \dots, y_n) \in \mathbb{R}^n$ is denoted as follows
\begin{equation*}
    f^{'}_{x_i} (y_1, y_2, \dots, y_n)
\end{equation*}
Moreover, partial derivative $f^{'}_{x_i}$ evaluated at point $(y_1, y_2, \dots, y_n)$ plus
partial derivative $f^{'}_{x_j}$ evaluated at point $(y_1, y_2, \dots, y_n)$
is equivalent to the sum of partial derivatives $f^{'}_{x_i}, \; f^{'}_{x_j}$
evaluated at point $(y_1, y_2, \dots, y_n)$ and to be denoted as
\begin{equation*}
    f^{'}_{x_i} (y_1, y_2, \dots, y_n) + f^{'}_{x_j} (y_1, y_2, \dots, y_n) = [f^{'}_{x_i} + f^{'}_{x_j}]
    (y_1, y_2, \dots, y_n)
\end{equation*}
Therefore, the following theorem in terms of partial derivatives
shows the relation between odd-power function $g_{y}$ and function $f_{y}$
\begin{thm}
    \label{thm:main-theorem}
    Let be a fixed point $v\in \mathbb{N}$, then ordinary derivative $g_v^{'}(u)$ of the odd-power function $g_v(x) = x^{2v + 1}$
    evaluate in point $u\in\mathbb{R}$ equals to partial derivative $(f_{v})^{'}_{x} (u, u)$ evaluate in point $(u, u)$ plus
    partial derivative $(f_{v})^{'}_{z} (u, u)$ evaluate in point $(u, u)$
    \begin{equation}
        g_v^{'} (u) = (f_{v})^{'}_{x} (u, u) + (f_{v})^{'}_{z} (u, u)
        \label{eq:odd-exponential-identity}
    \end{equation}
\end{thm}
In particular, it follows that for every pair $u \in \mathbb{R}, v \in \mathbb{N}$ an identity holds
\begin{align*}
(2v+1)
    u^{2v} &= (f_{v})^{'}_{x} (u, u) + (f_{v})^{'}_{z} (u, u) \\
    &= [(f_{v})^{'}_{x} + (f_{v})^{'}_{z}](u,u)
\end{align*}
that is also an ordinary derivative of odd-power function $t^{2v+1}$ evaluate in point $u\in\mathbb{R}$, therefore
\begin{align*}
    \frac{d}{dt} t^{2v+1} (u) &= (f_{v})^{'}_{x} (u, u) + (f_{v})^{'}_{z} (u, u) \\
    &= [(f_{v})^{'}_{x} + (f_{v})^{'}_{z}](u,u)
\end{align*}
To summarize and clarify all above, we provide a few examples that show
an application of the identity~\eqref{eq:odd-exponential-identity}.
\begin{example}
    \normalfont
    Identity~\eqref{eq:odd-exponential-identity} example for $x\in\mathbb{R}, \; z\in \mathbb{R}$ and $y=1$.
    Consider the explicit form of the function $f_{1} (x, z)$ i.e.,
    \[
        f_1(x, z) = 3 x z - 3 z^2 + 3 x z^2 - 2 z^3
    \]
    Therefore, derivative of $f_{1}$ with respect to $x$ equals to
    \[
        (f_1)^{'}_{x} = \lim_{d \to 0} \frac{3 d z + 3 d z^2}{d} = 3 z + 3 z^2
    \]
    Consider derivative of the function $f_1$ with respect to $z$, that is
    \begin{align*}
    (f_1)
        ^{'}_{z}
        &= \lim_{d \to 0} \left[\frac{-3 d^2 - 2 d^3 + 3 d x + 3 d^2 x - 6 d z - 6 d^2 z + 6 d x z - 6 d z^2}{d} \right] \\
        &= \lim_{d \to 0} \left[ -3 d - 2 d^2 + 3 x + 3 d x - 6 z - 6 d z + 6 x z - 6 z^2 \right] \\
        &=3 x - 6 z + 6 x z - 6 z^2
    \end{align*}
    Summing up both partial derivatives $(f_1)^{'}_{x} (u, u)$ and $(f_1)^{'}_{z} (u, u)$ evaluate in point $(u, u)$, we get
    \begin{align*}
    (f_1) ^{'}_{x} + (f_1)^{'}_{z} = 3 x - 3 z + 6 x z - 3 z^2
    \end{align*}
    Substituting $x=z=u$ yields
    \begin{align*}
        \frac{d}{dt} t^{3} (u) = [(f_1)^{'}_{x} + (f_1)^{'}_{z}] (u,u)  = 3 u^2
    \end{align*}
    that confirms the results of the theorem~\ref{thm:main-theorem}.
\end{example}
\begin{example}
    \normalfont
    Identity~\eqref{eq:odd-exponential-identity} example for $x\in\mathbb{R}, \; z\in \mathbb{R}$ and $y=2$.
    Consider the explicit form of the function $f_{2} (x, z)$ i.e.,
    \[
        f_2 (x, z) = 5 x^2 z - 15 x z^2 + 15 x^2 z^2 + 10 z^3 - 30 x z^3 + 10 x^2 z^3 + 15 z^4 - 15 x z^4 + 6 z^5
    \]
    Therefore, derivative of $f_{2}$ with respect to $x$ equals to
    \begin{align*}
    (f_2)
        ^{'}_{x} &= \lim_{d \to 0} \left[ 5 d z + 10 x z - 15 z^2 + 15 d z^2 + 30 x z^2 - 30 z^3 + 10 d z^3 +
        20 x z^3 - 15 z^4 \right] \\
        &= 10 x z - 15 z^2 + 30 x z^2 - 30 z^3 + 20 x z^3 - 15 z^4
    \end{align*}
    Consider derivative of the function $f_2$ with respect to $z$, that is
    \begin{align*}
    (f_2)
        ^{'}_{z}
        &= 5 x^2 - 30 x z + 30 x^2 z + 30 z^2 - 90 x z^2 + 30 x^2 z^2 + 60 z^3 - 60 x z^3 + 30 z^4
    \end{align*}
    Combining both $(f_2)^{'}_{x} (x, z)$ and $(f_2)^{'}_{z} (x, z)$ evaluated at the point $(u, u)$ we get
    \begin{align*}
    (f_2)
        ^{'}_{x} + (f_2)^{'}_{z} &= 5 x^2 - 20 x z + 30 x^2 z + 15 z^2 - 60 x z^2 + 30 x^2 z^2 + 30 z^3 - 40 x z^3 + 15 z^4\\
        \frac{d}{dt} t^{5} (u) &= [(f_2)^{'}_{x} + (f_2)^{'}_{z}] (u,u) = 5 u^4
    \end{align*}
    that confirms the results of the theorem~\ref{thm:main-theorem}.
\end{example}
\begin{example}
    \normalfont
    Identity~\eqref{eq:odd-exponential-identity} example for $x\in\mathbb{R}, \; z\in \mathbb{R}$ and $y=3$.
    Consider the explicit form of the function $f_{3} (x, z)$ i.e
    \begin{align*}
        f_3 (x, z) &= -7 x z + 14 x^2 z + 7 z^2 - 42 x z^2 + 35 x^3 z^2 + 28 z^3 -140 x^2 z^3 + 70 x^3 z^3 + 175 x z^4 \\
        &- 210 x^2 z^4 + 35 x^3 z^4 -70 z^5 + 210 x z^5 - 84 x^2 z^5 - 70 z^6 + 70 x z^6 - 20 z^7
    \end{align*}
    Therefore, derivative of $f_{3}$ with respect to $x$ equals to
    \begin{align*}
    (f_3)
        ^{'}_{x} &= -7 z + 28 x z - 42 z^2 + 105 x^2 z^2 - 280 x z^3 + 210 x^2 z^3 + 175 z^4 - 420 x z^4 \\
        &+ 105 x^2 z^4 + 210 z^5 - 168 x z^5 + 70 z^6
    \end{align*}
    Consider derivative of the function $f_2$ with respect to $z$, that is
    \begin{align*}
    (f_3)
        ^{'}_{z} &= -7 x + 14 x^2 + 14 z - 84 x z + 70 x^3 z + 84 z^2 - 420 x^2 z^2 + 210 x^3 z^2 + 700 x z^3 \\
        &- 840 x^2 z^3 + 140 x^3 z^3 - 350 z^4 + 1050 x z^4 - 420 x^2 z^4 - 420 z^5 + 420 x z^5 - 140 z^6
    \end{align*}
    Combining both $(f_3)^{'}_{x} (x, z)$ and $(f_3)^{'}_{z} (x, z)$ evaluated at the point $(u, u)$ we get
    \begin{align*}
    (f_3)
        ^{'}_{x} + (f_3)^{'}_{z} &= -7 x + 14 x^2 + 7 z - 56 x z + 70 x^3 z + 42 z^2 - 315 x^2 z^2 + 210 x^3 z^2 \\
        &+ 420 x z^3 - 630 x^2 z^3 + 140 x^3 z^3 - 175 z^4 + 630 x z^4 - 315 x^2 z^4 - 210 z^5 \\
        &+ 252 x z^5 - 70 z^6 \\
        \frac{d}{dt} t^{3} (u) &= [(f_3) ^{'}_{x} + (f_3)^{'}_{z}] (u,u) = 7 u^6
    \end{align*}
    that confirms the results of the theorem~\ref{thm:main-theorem}.
\end{example}