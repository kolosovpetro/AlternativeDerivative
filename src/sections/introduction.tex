This manuscript provides an exponential identity in terms of partial derivatives,
extending the main idea explained in~\cite{kolosov_2022} that gives polynomial identity in a form as follows
\begin{equation}
    n^{2m+1} = \sum_{k=1}^{n} \sum_{r=0}^{m} \coeffA{m}{r} k^r (n-k)^r, \quad (m,n) \in \mathbb{N},
    \label{eq:odd-power-identity}
\end{equation}
where $\coeffA{m}{r}$ are real coefficients defined recursively, see~\cite{kolosov2016link}.
Define the function $f$ such that based on the identity~\eqref{eq:odd-power-identity} with the only difference that
values of $n, m$ in its left part appear to be parameters of the function $f$, that is
\begin{definition}
    \begin{equation}
        f(x, y, z) = \sum_{k=1}^{z} \sum_{r=0}^{y} \coeffA{y}{r} k^r (x-k)^r\label{eq:definition-f}
    \end{equation}
\end{definition}
Important to note that upper bound of the sum $\sum_{k=1}^{z}$ is parameter of the function $f$ in contrast
to the equation~\eqref{eq:odd-power-identity} where upper bound of the sum is $n$.